%Librerías a usar---------------------------------------------------------------------
\documentclass[11pt, twocolumn]{llncs}
\usepackage[spanish]{babel}
\usepackage[utf8]{inputenc}
\usepackage{multicol}
\usepackage{geometry}
\usepackage{graphicx}
\usepackage{wrapfig}
\usepackage{listings}
\usepackage{hyperref}
\usepackage{tabularx,booktabs,caption}
\newcolumntype{Z}{>{\raggedright}X}
\renewcommand\spanishtablename{Tabla}
\geometry{top=1in, left=1in, right=1in}

%Inicio del documento-----------------------------------------------------------------
\begin{document}
\twocolumn[

%Título y autores---------------------------------------------------------------------
\title{TIEMPO DE EJECUCIÓN DE UN ALGORITMO DE MATRICES CUADRADAS}
\author{
Germán Contreras  \and 
Luciano Grandi  \and
Luis Correa \and 
Rodrigo Aguirre  \\ 
 \email{\{usuario1\}\{usuario2\}\{usuario3\}\{usuario4\} @utem.cl}
}
\institute{Universidad Tecnológica Metropolitana del Estado de Chile, Santiago, Chile
}
\maketitle
\begin{@twocolumnfalse}

%Resumen del documento----------------------------------------------------------------
\begin{abstract}
Breve resumen.

%Palabras Clave-----------------------------------------------------------------------
\textit{Key words:} Datos, paper

\end{abstract}
\end{@twocolumnfalse}
]

%INTRODUCCIÓN-------------------------------------------------------------------------
\section{INTRODUCCIÓN}


%CONTEXTO Y PROPÓSITO DEL EXPERIMENTO-------------------------------------------------
\section{CONTEXTO Y PROPÓSITO}\label{contexto_proposito}

Un algoritmo es un conjunto de instrucciones definidas, o secuencias de pasos lógicos que permiten solucionar un problema o realizar una tarea determinada. Estos algoritmos pueden clasificarse acorde a su complejidad, ya sean sencillos o complejos. Al disponer de un algoritmo que funcione correctamente, se nos hace útil determinar ciertos criterios para cuantificar su comportamiento, o en otras palabras, su rendimiento.

Muy seguido se piensa que al hablar de algoritmos sencillos, estos no son competentes. Pero, a la hora de modelar un algoritmo, la sencillez es un componente muy interesante, dado que agiliza el estudio de su eficiencia y su mantenimiento. Cuando hablamos de rendimiento en un algoritmo, este se mide respecto a dos parámetros: el espacio de memoria y el tiempo de ejecución.

Se denomina como tiempo de ejecución de un algoritmo al rango de tiempo t en que un algoritmo comienza su proceso de ejecución hasta el término de este. Este tiempo dependerá de varios factores, tales como la calidad del código o la complejidad del mismo, y los datos de entrada.

Los datos de entrada (dependiendo del tamaño) al interactuar en un algoritmo, tendrán distintos comportamientos y tiempos de ejecución, tal como muchas funciones matemáticas que existen. Esta comparación no es trivial, dado que al saber la complejidad de una función matemática, podemos relacionarla con algún algoritmo y determinar de forma teórica su grado de dificultad. Este grado de dificultad se puede usar para determinar a qué orden de complejidad corresponde cada algoritmo, y así encontrar el que mejor se adapte al problema a solucionar.

Para tal efecto, se desarrollará un algoritmo que pueda crear y multiplicar dos matrices cuadradas de tamaño n. Una matriz es un arreglo bidimensional con un conjunto de datos, ordenados en filas ($i$) y columnas ($j$). Existen diversos tipos de matrices, como las matrices nulas, donde sus datos son solo ceros o las matrices cuadradas como se muestra en la \textit{Fig. 1}, las cuales tienen el mismo número de filas y columnas tal que \textit{n = i = j}.

\begin{figure}
\caption{\label{fig:matrices}Ejemplo de matrices cuadradas.}
\centering
\includegraphics[width=0.3\textwidth]{matrices.png}
\end{figure}

El uso de matrices cuadradas no solo está ligado a la programación o al álgebra clásica, sino que también a la economía en operaciones muy complejas como las descomposiciones de Cholesky y LU. En el ámbito computacional, las matrices son usadas dado su fácil manejo y liviandad para la manipulación de información. Bajo este contexto, son una buena composición para representar diversos tipos de grafos.

La implementación del algoritmo busca formular una función que pueda describir de manera teórica y matemática el comportamiento del tiempo de ejecución en base al tamaño de datos de entrada $n$, por medio de un análisis empírico respecto a la ejecución del algoritmo con distintos valores de entrada de $n$.

Se busca que en base a los valores obtenidos mediante el análisis empírico, estos sean comparados con valores obtenidos en base al modelo matemático formulado, con los mismos valores de entrada $n$ usados anteriormente. Con esto, se espera encontrar un margen de error RDP (\textit{Residual Prediction Deviation}) por cada valor de $n$ usado y explicar los valores RDP que sean interesantes.

El algoritmo en estudio está escrito en lenguaje de programación $C++$. Este algoritmo creará dos matrices cuadradas $A$ y $B$ de tamaño $n$, las cuales estarán pobladas con números enteros positivos aleatorios dentro de un rango definido (desde 1 hasta 100). Con las matrices pobladas, se procederán a multiplicar para formar una nueva matriz cuadrada $C$.

El código del algoritmo es el siguiente:

\end{document}